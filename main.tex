\iffalse
	前言:
		参考自一个祖传的模板,以及https://GitHub.com/LeyuDame/BNUCV/tree/main 上的BNU的latex简历模板中的代码。
		注意要用XeLaTeX编译链进行编译,且要进行三次编译才能显示照片。问就是LaTeX的锅。
		vscode+latex的话,配置的json中"latex-workshop.latex.recipes"添加:
		{
            "name": "XeLaTeX*3",
            "tools": [
                "xelatex",
                "xelatex",
                "xelatex"
            ]
        }
		即可使用三次XeLaTeX编译。

		LaTeX+VScode怎么配置看https://www.zhihu.com/column/p/166523064。

        每个章节的格式都能混着用,顺序都可以变,只是给了个例子。
        比如你找工作,可以把技能那部分往前挪。
        比如你竞赛经历很多,你就往前挪。
        比如你觉得“其他”有点多余,就删了。

        主要贡献还是把原本word的那个模板页眉页脚和背景完美加进来了。
        LaTeX排版就是很整齐,强迫症狂喜。


        记得要三次XeLaTeX编译!!!
        记得要三次XeLaTeX编译!!!
        记得要三次XeLaTeX编译!!!
        这个很重要,所以说三遍!

        你可以试试两次的,好像某些环境两次编译也能显示图像,但最好还是三次编译。
\fi

\documentclass[11pt]{article}
\usepackage{xltxtra}
\usepackage{bookmark}
\usepackage{hyperref}
\hypersetup{hidelinks}
\usepackage{url}
\urlstyle{tt}
\usepackage{multicol}
\usepackage{xcolor}
\usepackage{calc}
\usepackage{graphicx}
\usepackage{tikz}
\usetikzlibrary{calc}
\usepackage{fontspec}
\usepackage{xeCJK}
\usepackage{relsize}
\usepackage{xspace}
\usepackage{fontawesome}
\usepackage{titlesec}
\usepackage{enumitem}
\usepackage{siunitx}
\usepackage{amssymb}
\usepackage{tabularx}
\usepackage{multicol}
\usepackage{fontspec}


% 一些小设置,参考自https://github.com/LeyuDame/BNUCV/tree/main
\CJKsetecglue{}							            % 取消中文字符与数字之间的间隔
\protected\def\Cpp{{C\nolinebreak[4]\hspace{-.05em}\raisebox{.28ex}{\relsize{-1}++}}\xspace}	% 这是个更好看的C++写法,你直接写C++的话,+号会很大,可以使用\Cpp来代替
\setlength{\parindent}{0pt}							% 取消全局段落缩进
\pagenumbering{gobble}								% 取消页码显示
%\setlist{noitemsep}									% 禁用列表中项目之间的额外垂直间距,但保留列表周围的间距
%\setlist{nosep}										% 禁用列表中项目之间的额外垂直间距及列表周围的间距
\setlist[itemize]{topsep=0em, leftmargin=*}		% 增加了itemize顶部间距
\setlist[enumerate]{topsep=0em, leftmargin=*}	% 增加了enumerate顶部间距

\titleformat{\section}					    % 将原标题前面的数字取消了
  {\LARGE\bfseries\raggedright} 		      % 字体改为LARGE,bold,左对齐
  {}{0em}                      			  % 可用于添加全局标题前缀
  {}                           			  % 可用于添加代码
  [{\color{NPU_Blue}\titlerule}]            % 标题下方加一条线
\titlespacing*{\section}{0cm}{*1.2}{*1.2}	% 标题左边留白,上方1.2倍,下方1.2倍

\titleformat{\subsection}				    % 将原二级标题前面的数字取消了
  {\large\bfseries\raggedright} 		      % 字体改为large,bold,左对齐
  {}{0em}                      			  % 可用于添加全局二级标题前缀
  {}                           			  % 可用于添加代码
  []
\titlespacing*{\subsection}{0cm}{*1.2}{*1.2}% 二级标题左边留白,上方1.2倍,下方1.2倍

% 页面大小与页边距,按需求调整
\usepackage[
	a4paper,
	left=1.2cm,
	right=1.2cm,
	top=1.5cm,
	bottom=1cm,
	nohead
]{geometry}

% 中文字符间距
\renewcommand{\CJKglue}{\hskip 0.05em}

% 英文字体
\setmainfont[
    Path=fonts/,
    Extension=.ttf,
    BoldFont=* Bold,
]{Microsoft Yahei}
% 中文字体
\setCJKmainfont[
    Path=fonts/,
    Extension=.ttf,
    BoldFont=* Bold,
]{Microsoft Yahei}

% 主题色
% 西工大蓝
\definecolor{NPU_Blue}{RGB}{0, 80, 158}

% 这里把表格的行间距调成1.2倍了
\renewcommand{\arraystretch}{1.2}
% 这里把正文的行间距调成1.2倍了
\linespread{1.2}

%%%%%%%%%%%%%%%%%%%%%%%%%%%%%%%%%%%%%%%%%%%%%%%%%%%%%%%%%%%%%%%%%%%%%%%%%%%%%%%%%%%%%%%%%%%%%%%%%%%%%%%%%%%%%%%%%%%%%%%%%%%%%%%%%%%%%%%%%%%%%%%%%%
%    !!!!!!!! 记得改这里 !!!!!!!!
%%%%%%%%%%%%%%%%%%%%%%%%%%%%%%%%%%%%%%%%%%%%%%%%%%%%%%%%%%%%%%%%%%%%%%%%%%%%%%%%%%%%%%%%%%%%%%%%%%%%%%%%%%%%%%%%%%%%%%%%%%%%%%%%%%%%%%%%%%%%%%%%%%
% 学院
\newcommand{\school}{计算机与人工智能学院}
% 也可以不写英语
%\newcommand{\school}{电子信息学院}
% 联系方式
\newcommand{\contact}
{
    \small              % 换了更小的字号
    % \footnotesize       % 这比上面的小一号
    \scriptsize         % 这比上面的再小一号
    \textcolor{white}
    {
        \faEnvelope \quad \href{1506674707@qq.com}{1506674707@qq.com}    % 邮箱,前面的超链接可以直达邮箱软件
        \hspace{4em}    % 这里可以调间距
       % \faWechat \quad xxxxxxxxxxxxx               % 微信
        \hspace{4em}    % 这里可以调间距
        \faPhone \quad 13619891389                 % 手机号
        \hspace{4em}    % 这里可以调间距
        \faGithub \quad \href{https://github.com/zhummq}{https://github.com/zhummq}         % github
    }
}



%%%%%%%%%%%%%%%%%%%%%%%%%%%%%%%%%%%%%%%%%%%%%%%%%%%%%%%%%%%%%%%%%%%%%%%%%%%%%%%%%%%%%%%%%%%%%%%%%%%%%%%%%%%%%%%%%%%%%%%%%%%%%%%%%%%%%%%%%%%%%%%%%%
%    !!!!!!!! 这里开始就是正文了 !!!!!!!!
%%%%%%%%%%%%%%%%%%%%%%%%%%%%%%%%%%%%%%%%%%%%%%%%%%%%%%%%%%%%%%%%%%%%%%%%%%%%%%%%%%%%%%%%%%%%%%%%%%%%%%%%%%%%%%%%%%%%%%%%%%%%%%%%%%%%%%%%%%%%%%%%%%
\begin{document}
	% 如果有多页简历,请把页眉页脚和背景复制粘贴到第二页的内容之前
	% 页眉,校徽,学院名
	\begin{tikzpicture}[remember picture, overlay]
		\node[anchor=north, inner sep=0pt](header) at (current page.north){
			\includegraphics[width=\paperwidth]{images/header.png}
		};
		\node[anchor=west](school_logo) at (header.west){
			\hspace{0.5cm}
			\includegraphics[width=0.25\textwidth]{images/images1.png}
		};
		\node[anchor=east](school_name) at(header.east){
			\textcolor{white}{\textbf{\school}}
			\hspace{0.5cm}
		};
	\end{tikzpicture}
	\vspace{-4em}

	% 页脚,联系方式
	\begin{tikzpicture}[remember picture, overlay]
		\node[anchor=south, inner sep=0pt](footer) at (current page.south){
			\includegraphics[width=\paperwidth]{images/footer.png}
		};
        % 联系方式
        \node[anchor=center] at(footer.center){\contact};
	\end{tikzpicture}
	
	% 背景
	\begin{tikzpicture}[remember picture, overlay]
		\node[opacity=0.1] at(current page.center){
			\includegraphics[width=0.7\paperwidth, keepaspectratio]{images/Zhengzhou.png}
		};
	\end{tikzpicture}

	% 个人信息
    \begin{figure}[h]
        % 左半边,信息,比例占行宽87%,可以自己调
        \begin{minipage}{\textwidth}
            \section{\makebox[\widthof{\faUser}][c]{\color{NPU_Blue}{\faUser}}\quad 个人信息}
            \begin{tabularx}{\linewidth}{p{\widthof{出生日期:}}Xp{\widthof{政治面貌:}}X}
                姓名: & 朱庆森 & 性别: & 男 \\
                出生日期: & 2006年10月08日 & 政治面貌: & 共青团员 \\
                    
                %% 想多加几行的话,就按上面的格式自行补充
                %% 想加粗的话\textbf{}
                %% 想多加几列的话,把\begin{tabularx}{\textwidth}{这里}的内容改一下,可以自己搜一下tabularx怎么用,也可以问gpt/文心一言/讯飞。
            \end{tabularx}
        \end{minipage}
        % 右半边,照片,比例占行宽12%,可以自己调
        % images/example_avatar.png 替换成你证件照的路径。
        %\begin{minipage}{0.12\textwidth}
         %   \includegraphics[width=\linewidth]{images/example_avatar.png}
        %\end{minipage}
        % 尽量留至少1%的间距,不然会换行
    \end{figure}


	% 教育背景
    	% \faGraduationCap这类\fa开头的都是font awesome里的logo,想换成其他logo的话,可以看一下附带的fontawsome.pdf,自行替换。
		% \section{\makebox[\widthof{     这里!    }][c]{\color{NPU_Blue}{     和这里!    }}\quad 标题}
	\section{\makebox[\widthof{\faGraduationCap}][c]{\color{NPU_Blue}{\faGraduationCap}}\quad 教育背景}
	\vspace{-1em}
    \begin{table}[h!]
        \begin{tabularx}{\textwidth}{XXp{\widthof{2023年9月 -- 2027年6月}}}
            郑州大学  & 软件工程(本科) & 2023年9月 -- 2027年6月\\
            % 这里哪个高就加粗哪个,哪个不想放就留白 =w= 比如你综测高GPA低,你就只放综测和综测排名就好。
        \end{tabularx}
    \end{table}
 \section{\makebox[\widthof{\faWrench}][c]{\color{NPU_Blue}{\faWrench}}\quad 技能特长}
    \vspace{0.3em}
    \begin{itemize}
    \item 熟练使用 Java,go 语言
    \item 阅读过 redis 低版本源码,熟悉 redis 基本架构,底层数据结构,如事件执行流程,渐进式hash,dict,zipList,quickList,skipList等
    \item 熟悉 Java内存模型,深入了解 ThreadLoacl底层原理。
    \item 了解 select,poll epoll 多路复用底层原理。
    \end{itemize}
    % 项目经历(找导师一般都看中这个),可以改成“科研经历”
        % \faGears 这是齿轮,适合机械类,我电信的也喜欢齿轮,就用这个了
        % \faFlask 这是烧瓶,适合生化类
        % \faLaptop 这是个笔记本电脑,适合计算机类
        % \faUsers 这是三个人,适合商科
    \section{\makebox[\widthof{\faGears}][c]{\color{NPU_Blue}{\faLaptop}}\quad 项目经历}
    \vspace{0.3em}
    % 小技巧,老师想看的重点加粗,比如商科类的一般更想看到数字,工科类的更想看到技术
    \subsection{Mit 5840项目\hfill }
    
   \hfill 2025年3月-2025年6月
    \begin{itemize}
        \item 实现论文 MapReduce,使用\textbf{临时文件加重命名}保证写入文件的原子性
        \item 实现 raft 算法的中的 \textbf{leader 选举,日志复制,持久化,快照}等,同时使用 \textbf{二分法} 优化日志回退,减少日志回退时间。
        \item 使用 raft 作为底层保证各个状态机的同步,使用 \textbf{全局唯一ID} 保证执行指令的线性一致。
    \end{itemize}
    %主要负责药物的\textbf{前期研发,中期亲自试药和后期的投毒计划部署}与主要执行人员。成功将著名侦探\textbf{工藤新一}变成小孩摸样。药物的研发成功让\textbf{150万人}收益,最终盈利\textbf{3000万元}。
    
    % 这是个工科类加粗的例子
    \vspace{1em}                % 这是换行用的
    \subsection{Bt和Http多线程下载器 \hfill }

      \textbf{\href{https://github.com/zhummq/goDownLoad}{项目Github地址}}\hfill 2024年12月-2025年1月
    \begin{itemize}
        \item 完成 torrent 文件的解析,和与其它 peer 的通信流程。
        \item 使用 \textbf{mannger-worker} 并发的与多个 peer 通信,完成多线程下载。
        \item 借助\textbf{断点续传协议},并发发送请求,实现 Http 的多线程下载
        \item 使用 \textbf{Mmap} 直接映射文件到内存,减少 write 系统调用,显著减少写入磁盘的时间。
    \end{itemize}
    %\textbf{(标题与内容由GPT乱编的,现实中应该没有这论文吧。。大概。)}\textbf{独自进行前期研究,实验验证和论文写作}。重点探索了将\textbf{多模态医学影像}(如MRI和CT)与\textbf{自然语言模型}相结合的方法,以提高脑肿瘤检测的准确性。成功将对比学习和Prompt Fine-Tuning技术应用于医学影像分析中,将模型在脑肿瘤检测任务中的性能提升到了\textbf{超越SOTA的水平},特别是在 one-shot 场景下取得了显著的改进。

    % 其他(也是看你想不想写)
    \section{\makebox[\widthof{\faInfo}][c]{\color{NPU_Blue}{\faInfo}}\quad 其他}
    \begin{itemize}
        \item 技术博客(嗯嗯好傅): \textbf{\href{https://blog.csdn.net/2301_81836829?spm=1010.2135.3001.5421}{CSDN}, \href{https://www.cnblogs.com/eehaofu}{博客园}}
        \item 喜欢深入探索技术底层的原理,最喜欢蒋炎岩老师的一句话 \textbf{计算机的世界没有魔法}
    \end{itemize}

\end{document}
